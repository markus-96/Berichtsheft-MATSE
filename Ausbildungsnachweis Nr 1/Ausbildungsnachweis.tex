\documentclass[a4paper]{report}
\usepackage{calc,pifont,eurosym,amsmath,wasysym,amssymb,amsfonts}
\usepackage[T1]{fontenc}
\usepackage[ngerman]{babel}
\usepackage{xcolor}
\usepackage[explicit]{titlesec}
\usepackage[normalem]{ulem}
\usepackage[skip=0cm plus 0cm,indent=0cm]{parskip}
\usepackage{fancyhdr}
\usepackage[vmargin=1.5cm,hmargin=2.5cm,includeheadfoot,head=1cm,headsep=0.5cm,foot=12pt,footskip=16pt]{geometry}
\usepackage{array,supertabular,hhline,colortbl,enumitem,lastpage,hyperref}
\usepackage{longtable}
\usepackage{framed}
\usepackage{listofitems}
\usepackage{lipsum}
\usepackage{blindtext}
\usepackage{tabularray}
\DefTblrTemplate{contfoot-text}{normal}{Fortsetzung auf der nächsten Seite}
\SetTblrTemplate{contfoot-text}{normal} 
\DefTblrTemplate{conthead-text}{normal}{Fortsetzung} 
\SetTblrTemplate{conthead-text}{normal}
\DefTblrTemplate{firsthead,middlehead,lasthead}{default}{}
\hypersetup{colorlinks=true,allcolors=black}
% Headings
% Outline numbering
\setcounter{secnumdepth}{0}
\makeatletter
\newcommand\arraybslash{\let\\\@arraycr}
\makeatother
% Pages
\fancypagestyle{Standard}{\fancyhf{}
	\fancyhead[L]{Ausbildung zur/zum Mathematisch-technischen Softwareentwickler/in (MATSE) an der RWTH Aachen, im Forschungszentrum Jülich und kooperierenden Unternehmen}
	\fancyfoot[R]{Seite \thepage{} von \pageref{LastPage}}
	\renewcommand\headrulewidth{0.75pt}
	\renewcommand\footrulewidth{0pt}
	\renewcommand\thepage{\arabic{page}}
}
\pagestyle{Standard}
\setlength{\skip\footins}{0.12cm}
\setlength\tabcolsep{1mm}
\renewcommand\arraystretch{1.3}

\title{Ausbildungsnachweis Nr}
\title{Berichtsheft}

\newcommand{\nein}{40}

\newcommand{\Vorname}{}		%Änderung hat keine Wirkung
\newcommand{\Nachname}{}		%Änderung hat keine Wirkung
\newcommand{\Monat}{}		%Änderung hat keine Wirkung
\newcommand{\Jahr}{}		%Änderung hat keine Wirkung
\newcommand{\Betreuer}{}		%Änderung hat keine Wirkung
\newcommand{\Institut}{}		%Änderung hat keine Wirkung
\newcommand{\Datum}{}		%Änderung hat keine Wirkung
\newcommand{\BerichtNr}{}		%Änderung hat keine Wirkung
\newcommand{\Bemerkungen}{}		%Änderung hat keine Wirkung
\newcommand{\contentBericht}{}		%Änderung hat keine Wirkung

%	#1	date to check
%	#2	array of dates
%	#3	what to print
\newcommand{\testdates}[3]{%
	\foreachitem\z\in#2{%
		\ifnum\z=#1%
			#3%
		\fi%
	}%
}

%	#1	date to check
%	#2	1-->vormittag, 2-->nachmittag
\newcommand{\FuelleAus}[2]{%
	\testdates{#1}{\TageKrank}{K}%
	\testdates{#1}{\TageUrlaub}{U}%
	\testdates{#1}{\TageVorlesung}{V}%
	\testdates{#1}{\TageBlockveranstaltung}{B}%
	\testdates{#1}{\TageDienstreise}{D}%
	\ifnum#2=1%
		\testdates{#1}{\VormittagsKrank}{K}%
		\testdates{#1}{\VormittagsUrlaub}{U}%
		\testdates{#1}{\VormittagsVorlesung}{V}%
		\testdates{#1}{\VormittagsBlockveranstaltung}{B}%
		\testdates{#1}{\VormittagsDienstreise}{D}%
	\fi%
	\ifnum#2=2%
		\testdates{#1}{\NachmittagsKrank}{K}%
		\testdates{#1}{\NachmittagsUrlaub}{U}%
		\testdates{#1}{\NachmittagsVorlesung}{V}%
		\testdates{#1}{\NachmittagsBlockveranstaltung}{B}%
		\testdates{#1}{\NachmittagsDienstreise}{D}%
	\fi%
	
}

\begin{document}
\renewcommand{\Vorname}{Max}
\renewcommand{\Nachname}{Mustermann}
\renewcommand{\Betreuer}{Super Betreuer}
\renewcommand{\Institut}{Intitut f. Super-MATSEs}
\renewcommand{\Datum}{\today}

%Seite 1, die Tabelle wann man wo war.
%Jeweils in die geschwungenen Klammern schreiben, an welchen Tages etwas zutrifft.
%War man zB am 4. und 5. des Monats und am 3. Nachmittags krank, trägt man bei \TageKrank: "{4,5}" und bei \VormittagsKrank: "{3}" ein.
%Trifft etwas garnicht zu, muss man etwas höheres als 31 eintragen, zB bei \TageDienstreise: "{40}"
%\readlist*\TageKrank		{4,5,6}
\readlist*\TageKrank		{\nein}
\readlist*\TageUrlaub		{\nein}
\readlist*\TageVorlesung	{\nein}
\readlist*\TageBlockveranstaltung{\nein}
\readlist*\TageDienstreise	{\nein}

\readlist*\VormittagsKrank		{\nein}
\readlist*\VormittagsUrlaub		{\nein}
\readlist*\VormittagsVorlesung	{\nein}
\readlist*\VormittagsBlockveranstaltung{\nein}
\readlist*\VormittagsDienstreise{\nein}

\readlist*\NachmittagsKrank		{\nein}
\readlist*\NachmittagsUrlaub	{\nein}
\readlist*\NachmittagsVorlesung	{\nein}
\readlist*\NachmittagsBlockveranstaltung{\nein}
\readlist*\NachmittagsDienstreise{\nein}

%Seite 1, die Tabelle für Noten.
%Man kann fünf Noten eintragen. Hat man zB Lineare Algebra 2 geschrieben und hat eine 4,0 mit der Bemerkung "vier null gewinnt", schreibt man hier "{Lineare Algebra 2,{4,0},vier null gewinnt}"
%\readlist*\NoteEins{Lineare Algebra 2,{4,0},vier null gewinnt}
\readlist*\NoteEins{,{},}
\readlist*\NoteZwei{,{},}
\readlist*\NoteDrei{,{},}
\readlist*\NoteVier{,{},}
\readlist*\NoteFuenf{,{},}

%Seite 1, die "Besonderen Bemerkungen"
%\renewcommand{\Bemerkungen}{Hier können Bemerkungen stehen.}
\renewcommand{\Bemerkungen}{}
\begin{center}
	\textbf{\uline{Ausbildungsnachweis}}
\end{center}
\addcontentsline{toc}{part}{Ausbildungsnachweis Nr. \BerichtNr}

\begin{supertabular}{m{7.8cm}m{0.1cm}m{7.8cm}}
	\centering \Nachname, \Vorname & & \centering\arraybslash \Monat, \Jahr\\
	\hhline{-~-}
	\centering Name, Vorname & & \centering\arraybslash Berichtszeitraum (Monat, Jahr)\\
	%
	\centering \Betreuer & & \centering\arraybslash \Institut\\
	\hhline{-~-}
	\centering Name Betreuer/in & &\centering\arraybslash Institut / Firma\\
	%
	\centering \Datum & &\\
	\hhline{-~~}
	\centering Datum & &\\
\end{supertabular}
\vspace{1cm}

\begin{center}
	\begin{tabular}{lll}
		V = Vorlesung / Übung &	B = Blockveranstaltung & U = Urlaub\\
		D = Dienstreise & K = Krank &\\
	\end{tabular}
\end{center}
\begin{longtblr}
	{
		hline{1-3}={0.75pt},
		hline{4,5}={1-30}{0.75pt},
		vline{1-31}={0.75pt},
		vline{32,33}={1,2}{0.75pt},
		colspec = {*{2}{X[c]} *{2}{X[l]} *{2}{X[l]} *{2}{X[l]} *{2}{X[l]} *{2}{X[l]} *{2}{X[l]} *{2}{X[l]} *{2}{X[l]} *{2}{X[l]} *{2}{X[l]} *{2}{X[l]} *{2}{X[l]} *{2}{X[l]} *{2}{X[l]} *{2}{X[l]}},
		row{1}  = {font=\bfseries, c},
		row{3}  = {font=\bfseries, c},
		rowhead=1
	}
	\SetCell[c=2]{c}1& & \SetCell[c=2]{c}2 & & \SetCell[c=2]{c}3 & & \SetCell[c=2]{c}4 & & \SetCell[c=2]{c}5 & & \SetCell[c=2]{c}6 & & \SetCell[c=2]{c}7 & & \SetCell[c=2]{c}8 & & \SetCell[c=2]{c}9 & & \SetCell[c=2]{c}10 & & \SetCell[c=2]{c}11 & & \SetCell[c=2]{c}12 & & \SetCell[c=2]{c}13 & & \SetCell[c=2]{c}14 & & \SetCell[c=2]{c}15 & & \SetCell[c=2]{c}16 \\
	\FuelleAus{1}{1}&\FuelleAus{1}{2}&
	\FuelleAus{2}{1}&\FuelleAus{2}{2}&
	\FuelleAus{3}{1}&\FuelleAus{3}{2}&
	\FuelleAus{4}{1}&\FuelleAus{4}{2}&
	\FuelleAus{5}{1}&\FuelleAus{5}{2}&
	\FuelleAus{6}{1}&\FuelleAus{6}{2}&
	\FuelleAus{7}{1}&\FuelleAus{7}{2}&
	\FuelleAus{8}{1}&\FuelleAus{8}{2}&
	\FuelleAus{9}{1}&\FuelleAus{9}{2}&
	\FuelleAus{10}{1}&\FuelleAus{10}{2}&
	\FuelleAus{11}{1}&\FuelleAus{11}{2}&
	\FuelleAus{12}{1}&\FuelleAus{12}{2}&
	\FuelleAus{13}{1}&\FuelleAus{13}{2}&
	\FuelleAus{14}{1}&\FuelleAus{14}{2}&
	\FuelleAus{15}{1}&\FuelleAus{15}{2}&
	\FuelleAus{16}{1}&\FuelleAus{16}{2}\\
	\SetCell[c=2]{c}17& & \SetCell[c=2]{c}18 & & \SetCell[c=2]{c}19 & & \SetCell[c=2]{c}20 & & \SetCell[c=2]{c}21 & & \SetCell[c=2]{c}22 & & \SetCell[c=2]{c}23 & & \SetCell[c=2]{c}24 & & \SetCell[c=2]{c}25 & & \SetCell[c=2]{c}26 & & \SetCell[c=2]{c}27 & & \SetCell[c=2]{c}28 & & \SetCell[c=2]{c}29 & & \SetCell[c=2]{c}30 & & \SetCell[c=2]{c}31 \\
	\FuelleAus{17}{1}&\FuelleAus{17}{2}&
	\FuelleAus{18}{1}&\FuelleAus{18}{2}&
	\FuelleAus{19}{1}&\FuelleAus{19}{2}&
	\FuelleAus{20}{1}&\FuelleAus{20}{2}&
	\FuelleAus{21}{1}&\FuelleAus{21}{2}&
	\FuelleAus{22}{1}&\FuelleAus{22}{2}&
	\FuelleAus{23}{1}&\FuelleAus{23}{2}&
	\FuelleAus{24}{1}&\FuelleAus{24}{2}&
	\FuelleAus{25}{1}&\FuelleAus{25}{2}&
	\FuelleAus{26}{1}&\FuelleAus{26}{2}&
	\FuelleAus{27}{1}&\FuelleAus{27}{2}&
	\FuelleAus{28}{1}&\FuelleAus{28}{2}&
	\FuelleAus{29}{1}&\FuelleAus{29}{2}&
	\FuelleAus{30}{1}&\FuelleAus{30}{2}&
	\FuelleAus{31}{1}&\FuelleAus{31}{2} \\
\end{longtblr}
\begin{center}
	\begin{tabular}{ll}
		1. Kästchen = Vormittag & 2. Kästchen = Nachmittag\\
	\end{tabular}
\end{center}

\textbf{In diesem Monat wurden folgende Klausuren geschrieben}
\begin{longtblr}
	{
		hlines={0.75pt},
		vlines={0.75pt},
		colspec = {*{2}{X[l]} *{2}{X[l]} *{2}{X[l]}},
		row{1}  = {font=\bfseries, c}
	}
	Fach: & Note (optional): & Bemerkung:\\
	\NoteEins[1]&\NoteEins[2]&\NoteEins[3]\\
	\NoteZwei[1]&\NoteZwei[2]&\NoteZwei[3]\\
	\NoteDrei[1]&\NoteDrei[2]&\NoteDrei[3]\\
	\NoteVier[1]&\NoteVier[2]&\NoteVier[3]\\
	\NoteFuenf[1]&\NoteFuenf[2]&\NoteFuenf[3]\\
\end{longtblr}

\begin{longtblr}
	{
		vlines,
		hlines,
		colspec = {*{2}{X[l]}},
		row{1}  = {font=\bfseries, c}
	}
	Besondere Bemerkungen\\
	\begin{minipage}[t][2.5cm][t]{\textwidth}
		\Bemerkungen
	\end{minipage}
\end{longtblr}
\vspace{0.5cm}

\textbf{Die Ausbildungsnachweise sind vom Auszubildenden bis zum Ende der Ausbildung aufzubewahren und bei der
Abschlussprüfung vorzulegen.}

\newpage
Bitte beachten Sie beim Ausfüllen das Merkblatt zum Ausbildungsnachweis.

\begin{framed}
	Detaillierte Beschreibung der Tätigkeit (ausgeführte Arbeiten) bzw. Themen der betrieblichen Berufsausbildung. Die Fähigkeiten können monatsbezogen zusammengefasst werden, sollen jedoch ermöglichen, die geleistete Arbeit nachvollziehen zu können.
	\noindent\rule[1ex]{\hsize}{1pt}
	Hallo
\section{Lorem Ipsum}
\subsection{Lorem}
	\lipsum[1-2]
\subsection{Ipsum}
	\lipsum[4]
\section{Blindtext}
	\blinditemize[5]
	\blindmathpaper
	
\end{framed}
%\end{tabular}
\end{document}
