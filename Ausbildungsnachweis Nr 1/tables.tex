%Seite 1, die Tabelle wann man wo war.
%Jeweils in die geschwungenen Klammern schreiben, an welchen Tages etwas zutrifft.
%War man zB am 4. und 5. des Monats und am 3. Nachmittags krank, trägt man bei \TageKrank: "{4,5}" und bei \VormittagsKrank: "{3}" ein.
%Trifft etwas garnicht zu, muss man etwas höheres als 31 eintragen, zB bei \TageDienstreise: "{40}"
%\readlist*\TageKrank		{4,5,6}
\readlist*\TageKrank		{\nein}
\readlist*\TageUrlaub		{\nein}
\readlist*\TageVorlesung	{\nein}
\readlist*\TageBlockveranstaltung{\nein}
\readlist*\TageDienstreise	{\nein}

\readlist*\VormittagsKrank		{\nein}
\readlist*\VormittagsUrlaub		{\nein}
\readlist*\VormittagsVorlesung	{\nein}
\readlist*\VormittagsBlockveranstaltung{\nein}
\readlist*\VormittagsDienstreise{\nein}

\readlist*\NachmittagsKrank		{\nein}
\readlist*\NachmittagsUrlaub	{\nein}
\readlist*\NachmittagsVorlesung	{\nein}
\readlist*\NachmittagsBlockveranstaltung{\nein}
\readlist*\NachmittagsDienstreise{\nein}

%Seite 1, die Tabelle für Noten.
%Man kann fünf Noten eintragen. Hat man zB Lineare Algebra 2 geschrieben und hat eine 4,0 mit der Bemerkung "vier null gewinnt", schreibt man hier "{Lineare Algebra 2,{4,0},vier null gewinnt}"
%\readlist*\NoteEins{Lineare Algebra 2,{4,0},vier null gewinnt}
\readlist*\NoteEins{,{},}
\readlist*\NoteZwei{,{},}
\readlist*\NoteDrei{,{},}
\readlist*\NoteVier{,{},}
\readlist*\NoteFuenf{,{},}

%Seite 1, die "Besonderen Bemerkungen"
%\renewcommand{\Bemerkungen}{Hier können Bemerkungen stehen.}
\renewcommand{\Bemerkungen}{}